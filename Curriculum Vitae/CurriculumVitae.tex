% Curriculum Vitae
% Kevin Uriel Manzano Rios

\documentclass[10pt,a4paper,ragged2e,withhyper]{altacv}

\geometry{left=1.25cm,right=1.25cm,top=1.5cm,bottom=1.5cm,columnsep=1.2cm}
\usepackage{paracol}
  \usepackage[default]{lato}

\definecolor{VividPurple}{HTML}{3E0097}
\definecolor{SlateGrey}{HTML}{2E2E2E}
\definecolor{LightGrey}{HTML}{666666}

\colorlet{heading}{VividPurple}
\colorlet{headingrule}{VividPurple}
\colorlet{accent}{VividPurple}
\colorlet{emphasis}{SlateGrey}
\colorlet{body}{LightGrey}

\renewcommand{\itemmarker}{{\small\textbullet}}
\renewcommand{\ratingmarker}{\faCircle}

\begin{document}
\name{Kevin Uriel Manzano Rios}
\personalinfo{
  \email{kmanzanor24@gmail.com}
  \phone{+(52) 56 1748 0590}
  \location{México}
  \linkedin{KevinUrielManzanoRios}
  \github{KevinUrielAdler}
}

\makecvheader
\AtBeginEnvironment{itemize}{\small}
\columnratio{0.6}
\begin{paracol}{2}
  \cvsection{Formación}
  \cvevent{Ingeniería en Inteligencia Artificial}{\href{https://es.wikipedia.org/wiki/Escuela_Superior_de_Cómputo}{Escuela Superior de Cómputo (\color{blue}ESCOM, IPN\color{black})}}{Ene 2022 -- Previsto Dic 2025}{Av. Juan de Dios Bátiz 46188, CDMX}
  \begin{itemize}
    \item Especialización prevista: Ciencias de la Computación
  \end{itemize}
  \divider
  \cvevent{{\href{https://drive.google.com/file/d/1U6rwow80xCz0LrepnSFyvGuc5V7Hr1Id/view?usp=sharing}{\color{blue}Carrera Técnica\color{black}}} en Informática}{{\href{https://www.cecyt13.ipn.mx}{Centro de Estudios Científicos y Tecnológicos N°13 (\color{blue}IPN\color{black})}}}{Ago 2018 -- Ago 2021}{Calz Taxqueña 1620, CDMX}
  \begin{itemize}
    \item {\href{https://drive.google.com/file/d/1osz7QmjWUH6OcZ8AOrqVXEIJf-Pet3Q5/view?usp=sharing}{\color{blue}Promedio general\color{black}}}: 82 de 100
  \end{itemize}
  \divider
  \cvevent{Cursos de Python 3 y Azure}{}{2021}{}
  \begin{itemize}
    \item Completé un curso oficial de Microsoft para certificarme en "{\href{https://drive.google.com/file/d/1iRQRtXmNAItFIWmfjOP1m_35C4pAPpTP/view?usp=sharing}{\color{blue}Microsoft Azure Fundamentals AZ 900\color{black}}}" y un curso de Python usando python 3, SQLite3, y Flask.
  \end{itemize}

  \cvsection{Premios y concursos}
  \cvevent{{\href{https://drive.google.com/file/d/1a9Y24AIVQ5aGS7OPvPD1qg4wDesXFmRi/view?usp=sharing}{\color{blue}Segundo Lugar\color{black}}} (Concurso estatal)}{Olimpiada Mexicana de Informática ({\href{https://www.olimpiadadeinformatica.org.mx/OMI/OMI/InfoGeneral/Que_es_la_OMI.aspx}{\color{blue}OMI\color{black}}})}{}{}
  \begin{itemize}
    \item Completé el curso impartido por el Centro de Innovación Tecnológica Educativa (CITE) sobre programación competitiva en C++.
    \item Participé en la Olimpiada Mexicana de Informática (OMI) y han obtenido la medalla de plata a nivel estatal.
  \end{itemize}
  \divider
  \cvevent{Concursos de Robótica}{Robot War, MakeX}{}{}
  \begin{itemize}
    \item Participé en 2 concursos de robótica, {\href{https://drive.google.com/file/d/1xETjexZx_X1Bg46y_-m8kKneEbaz48dQ/view?usp=sharing}{\color{blue}guerra de robots\color{black}}} y {\href{https://drive.google.com/file/d/1QcXzJbDVQ1dSvtQrdd_j387KvKlfXH8J/view?usp=sharing}{\color{blue}MakeX Robotics Competition\color{black}}}, patrocinados por MISUMI y CreativaKids respectivamente.
  \end{itemize}
  \divider
  \cvevent{Competición Internacional Universitaria de Programación}{{\href{https://es.wikipedia.org/wiki/Competición_Internacional_Universitaria_de_Programación}{\color{blue}ICPC\color{black}}}}{}{}
  \begin{itemize}
    \item Participé en el ICPC en 2020, {\href{https://drive.google.com/file/d/1OwYn_YItOoTw_DI-Dajlrwc_oC57Rv6G/view?usp=sharing}{\color{blue}nuestro equipo\color{black}}} llegó al top 100 en la final.
  \end{itemize}

  \switchcolumn
  \cvsection{Experiencia}

  \cvachievement{\faUsers}{Profesor en la Vocacional}{Como parte de mi servicio social enseñé C/C++, algoritmos, y resolución de problemas a 60 estudiantes (480 horas) y enseñé a utilizar el OMIbot a estudiantes interesados en la OMI.}
  \divider

  \cvachievement{\faBook}{Proyecto de Algorithm Learning}
  {Como proyecto final de bachillerato realicé un {\href{https://github.com/KevinUrielAdler/Allearning}{\color{blue}programa\color{black}}} llamado "Allearning" para enseñar programación competitiva en C++, además de una {\href{https://github.com/KevinUrielAdler/Allearning-Web}{\color{blue}página web\color{black}}} como complemento.}
  \divider

  \cvachievement{\faLaptopCode}{Calculadora de Máximos y Mínimos}{Como proyecto en segundo semestre de universidad creé un {\href{https://github.com/KevinUrielAdler/Maximos-y-Minimos}{\color{blue}programa\color{black}}} para calcular máximos y mínimos de funciones escalares de variable vectorial usando SymPy.}

  \cvsection{Tecnologías}
  \cvtag{C/C++ (4 años)}
  \cvtag{C\# (3 años)}\\
  \cvtag{Python (3 años)}
  \cvtag{Git/GitHub (2 años)}\\
  \cvtag{MySQL (2 años)}
  \cvtag{LaTeX (1 año)}

  \cvsection{Aptitudes}
  \cvtag{Trabajo en Equipo}
  \cvtag{Arquitectura de Software}\\
  \cvtag{Optimización de Código}
  \cvtag{Enfoque Lógico}
  \cvtag{Iniciativa}
  \cvtag{Resolución de Problemas}
  \cvtag{Creatividad}

  \newpage
\end{paracol}
\end{document}
