% Resume
% Kevin Uriel Manzano Rios

\documentclass[10pt,a4paper,ragged2e,withhyper]{altacv}

\geometry{left=1.25cm,right=1.25cm,top=1.5cm,bottom=1.5cm,columnsep=1.2cm}
\usepackage{paracol}
\usepackage[default]{lato}

\definecolor{VividPurple}{HTML}{3E0097}
\definecolor{SlateGrey}{HTML}{2E2E2E}
\definecolor{LightGrey}{HTML}{666666}

\colorlet{heading}{VividPurple}
\colorlet{headingrule}{VividPurple}
\colorlet{accent}{VividPurple}
\colorlet{emphasis}{SlateGrey}
\colorlet{body}{LightGrey}

\renewcommand{\itemmarker}{{\small\textbullet}}
\renewcommand{\ratingmarker}{\faCircle}
\renewcommand{\cvDateMarker}{\faCalendar*[regular]}
\renewcommand{\cvLocationMarker}{\faMapMarker*}

\begin{document}
\name{Kevin Uriel Manzano Rios}
\personalinfo{
  \email{kmanzanor24@gmail.com}
  \phone{+(52) 56 1748 0590}
  % \homepage{https://github.com/KevinUrielAdler}
  \github{KevinUrielAdler}
  \linkedin{KevinUrielManzanoRios}
  \location{México}
}

\makecvheader
\AtBeginEnvironment{itemize}{\small}
\columnratio{0.6}
\begin{paracol}{2}

  \cvsection{Experience}

  \cvachievement{\faLaptopCode}{Python Developer and Natural Language Processing
  at Rombo Works}{Led the information retrieval team in developing a RAG
  (Retrieval-Augmented Generation) system to power an advanced chatbot
  capable of answering natural language queries about the Official Gazette of the
  Federation of Mexico. Implemented intelligent retrieval solutions with Python,
  leveraging vector databases and natural language APIs, achieving
  complete and structured responses, regardless of the query period or type.}
  {Jan 2024 -- Oct 2024}
  \divider

  \cvachievement{\faChalkboardTeacher}{Teacher in High School}
  {Developed and delivered C/C++, algorithms, and problem solving lessons to 60
  students in the Algorithm Club and
  taught how to use the OMIbot to students interested in the Mexican Olympiad
  in Informatics (OMI).}{Aug 2020 -- Aug 2021}

  \cvsection{Relevant Projects}

  \cvachievementalt{\faBook}{Algorithm Learning Project}
  {Developed {\href{https://github.com/KevinUrielAdler/Allearning}{\color{blue}"Allearning"\color{black}}},
  a platform to teach competitive programming in C++. The application offers users
  interactive lessons and exercises. To complement the experience and make it
  accessible from any device, created a
  {\href{https://github.com/KevinUrielAdler/Allearning-Web}{\color{blue}
  web page\color{black}}} using .NET.}
  \divider

  \cvachievementalt{\faChild}{Conversational Virtual Assistant “ZAID}
  {Created {\href{https://github.com/KevinUrielAdler/AvZ}{\color{blue}
  ZAID\color{black}}}, a conversational virtual assistant using Python.
  Implemented its ability to interact with the user in natural language and
  perform various automated tasks. The system was designed to run in the
  background and respond to voice commands intelligently.}

  \cvsection{Education}
  \cvevent{B.S. in Artificial Intelligence Engineering}{\href{https://en.wikipedia.org/wiki/ESCOM}{Superior School of Computer Sciences (\color{blue}ESCOM, IPN\color{black})}}{Jan 2022 -- Expected Dec 2025}{Av. Juan de Dios Bátiz 46188, CDMX}
  \begin{itemize}
    \item Intended major: Computer Science
  \end{itemize}
  \divider

  \cvevent{{\href{https://drive.google.com/file/d/1U6rwow80xCz0LrepnSFyvGuc5V7Hr1Id/view?usp=sharing}{\color{blue}Technical Career\color{black}}} in Informatics}{{\href{https://en.wikipedia.org/wiki/CECyT}{Center of Scientific and Technological Studies N°13 (\color{blue}IPN\color{black})}}}{Aug 2018 -- Aug 2021}{Calz Taxqueña 1620, CDMX}
  \begin{itemize}
    \item {\href{https://drive.google.com/file/d/1osz7QmjWUH6OcZ8AOrqVXEIJf-Pet3Q5/view?usp=sharing}{\color{blue}Overall Average\color{black}}}: 82 out of 100
  \end{itemize}
  \divider

  \cvevent{Python 3 and Azure courses}{}{2021}{}
  \begin{itemize}
    \item Took an official Microsoft course to get certified in "{\href{https://drive.google.com/file/d/1iRQRtXmNAItFIWmfjOP1m_35C4pAPpTP/view?usp=sharing}{\color{blue}Microsoft Azure Fundamentals AZ 900\color{black}}}" and a Python course using python 3, SQLite3, and Flask.
  \end{itemize}

  \switchcolumn

  \cvsection{Tecnologies}
  \cvtag{C/C++ (6 years)}
  \cvtag{Python (5 Years)}\\
  \cvtag{TensorFlow (2 years)}
  \cvtag{PyTorch (1 year)}\\
  \cvtag{C\# (3 years)}
  \cvtag{Docker (1 year)}
  \cvtag{Linux (2 years)}\\
  \cvtag{HTML (3 years)}
  \cvtag{MySQL/SQLite (4 years)}\\
  \cvtag{Chroma DB (1 year)}
  \cvtag{Git/GitHub (5 year)}

  \cvsection{Aptitudes}
  \cvtag{Teamwork}
  \cvtag{Software Architecture}\\
  \cvtag{Code Optimization}
  \cvtag{Logical Approach}
  \cvtag{Initiative}
  \cvtag{Problem Solving}
  \cvtag{Creativity}

  \cvsection{Languages}
  \begin{tabular}{l | ll}
    \textbf{Spanish} & Native language\\
    \textbf{English} & \href{https://drive.google.com/file/d/1gBM0AOUTz7h87_4FS4IsUOT8kphNWD5m/view?usp=sharing}{\color{blue}B2 - Upper Intermediate (CEFR)\color{black}}\\
    \textbf{German} & A2 - Elementary (CEFR)
  \end{tabular}

  \cvsection{Awards and Contests}
  \cvevent{{\href{https://drive.google.com/file/d/1a9Y24AIVQ5aGS7OPvPD1qg4wDesXFmRi/view?usp=sharing}{\color{blue}Second Place\color{black}}} (State competition)}{Mexican Olympiad in Informatics ({\href{https://www.olimpiadadeinformatica.org.mx/OMI/OMI/InfoGeneral/Que_es_la_OMI.aspx}{\color{blue}OMI\color{black}}})}{}{}
  \begin{itemize}
    \item Completed a preparatory course given by the Center for Educational Technological Innovation (CITE) on competitive programming in C++.
    \item Participated in the Mexican Olympiad in Informatics (OMI) and won the silver medal at the state level.
  \end{itemize}
  \divider
  \cvevent{Robotics Contests}{Robot War, MakeX}{}{}
  \begin{itemize}
    \item Have participated in 2 robotics contests, {\href{https://drive.google.com/file/d/1xETjexZx_X1Bg46y_-m8kKneEbaz48dQ/view?usp=sharing}{\color{blue}robot war\color{black}}} and {\href{https://drive.google.com/file/d/1QcXzJbDVQ1dSvtQrdd_j387KvKlfXH8J/view?usp=sharing}{\color{blue}MakeX Robotics Competition\color{black}}}, sponsored by MISUMI and CreativaKids respectively.
  \end{itemize}
  \divider
  \cvevent{International Collegiate Programming Contest}{{\href{https://en.wikipedia.org/wiki/International_Collegiate_Programming_Contest}{\color{blue}ICPC\color{black}}}}{}{}
  \begin{itemize}
    \item Have participated in the ICPC in 2020, {\href{https://drive.google.com/file/d/1OwYn_YItOoTw_DI-Dajlrwc_oC57Rv6G/view?usp=sharing}{\color{blue}our team\color{black}}} reached the top 100 in the finals.
  \end{itemize}

  \newpage
\end{paracol}
\end{document}
